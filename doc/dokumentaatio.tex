\documentclass[a4paper]{scrartcl}
\usepackage[finnish]{babel}
\usepackage[utf8]{inputenc}
\usepackage[T1]{fontenc}
\usepackage{mathptmx}
\usepackage[numbers]{natbib}
\usepackage{hyperref}
\usepackage{tikz-uml}
\usepackage{enumitem}
\usepackage{calc}
\setlist[description]{font={\rmfamily}}
\addtokomafont{disposition}{\rmfamily}

\subject{Aineopintojen harjoitustyö: Tietokantasovellus}
\author{Jarno Leppänen}
\title{Divvd -- Useilla valuutoilla toimiva kustannusten jako}
\subtitle{Dokumentaatio}
\date{16.3.2014}

\begin{document}

\maketitle

\section{Johdanto}

Kustannusten jako on tavallinen ongelma esimerkiksi illanvietossa ystävien
kanssa tai taloudenpidossa puolison tai kämppäkavereiden välillä. Erityisen
hankalaksi tehtävä osoittautuu, jos jaettavissa kustannuksissa on käytetty on
useita eri valuuttoja. Tällainen tilanne voi syntyä esimerkiksi pitkän
reppumatkan aikana.

Työssä toteutetaan kustannusten jakoon tarkoitettu kirjanpitojärjestelmä, jossa
kirjauksia voi tehdä eri valuutoilla. Käyttäjä voi rekisteröitymisen ja
kirjautumisen jälkeen luoda \textit{tilikirjoja}, jotka ovat kokoelmia
käyttäjän syöttämistä \textit{tilitapahtumista}. Tilitapahtumaan kirjataan
tapahtuman osapuolet sekä näiden tulot ja menot tapahtumaan liittyen.
Järjestelmä laskee tilikirjassa olevien osapuolten kokonaissaldot haluttuun
valuuttaan muunnettuna ja osaa lisäksi ehdottaa tapoja velkojen
takaisinmaksuun. Takaisinmaksujärjestely voidaan ratkaista esimerkiksi
minimoimalla järjestelyssä tapahtuvien transaktioiden
lukumäärä\cite{verhoeff2004settling}.

\subsection{Toimintaympäristö}

Ohjelma toteutetaan web-sovelluksena heroku-PaaS-palvelun päälle.
Sovelluskehyksenä käytetään herokun tukemaa node.js-kehystä. Ohjelmointikielenä
on Javascript sekä selain-, että palvelinympäristössä.

Tietokantana toimii herokun tarjoama PostgreSQL. Tietokantaa käsitellään
suoraan Postgre\-SQL:n laajennetun SQL-kielen avulla. Toteutuksessa tullaan
hyödyntämään SQL-standardiin kuulumattomia trigger- ja view-mekanismeja, joten
tietokantamoduli tulee olemaan sidottu Post\-gre\-SQL-jär\-jes\-tel\-mään.

\section{Yleiskuva järjestelmästä}

\subsection{Käyttötapauskaavio}

\newcommand{\coldist}{5}
\begin{tikzpicture}
\begin{umlsystem}[x=5,fill=yellow!20]{Divvd-järjestelmä}
  \umlusecase[name=register]{rekisteröityminen}
  \umlusecase[y=-2,name=login]{kirjautuminen}
  \umlusecase[y=-4,name=ledger,width=1.5cm]{tilikirjojen käsittely}
  \umlusecase[y=-6,name=transaction,width=2.2cm]{tilitapahtumien käsittely}
  \umlusecase[y=-10,name=totals,width=2cm]{loppusaldojen tulostus}
  \umlusecase[y=-12,name=settle,width=2.9cm]{maksusuunnitelman tulostus}
  \umlusecase[y=-8,name=currency,width=2.4cm]{valuuttakurssien muokkaus}
  \umlusecase[y=-4,x=\coldist,name=add-ledger,width=1.5cm]{tilikirjan lisäys}
  \umlusecase[y=-2,x=\coldist,name=delete-ledger,width=1.5cm]{tilikirjan poisto}
  \umlusecase[y=-6,x=\coldist,name=add-transaction,width=2.2cm]{tilitapahtuman
    lisäys}
  \umlusecase[y=-10,x=\coldist,name=delete-transaction,width=2.2cm]{tilitapahtuman
    poisto}
  \umlusecase[y=-8,x=\coldist,name=edit-transaction,width=2.2cm]{tilitapahtuman
    muokkaus}
\end{umlsystem}
\umlactor[y=-6,x=-1,name=user]{käyttäjä}
\umlassoc{user}{register}
\umlassoc{user}{login}
\umlassoc{user}{ledger}
\umlassoc{user}{transaction}
\umlassoc{user}{totals}
\umlassoc{user}{settle}
\umlassoc{user}{currency}
\umlinherit{add-ledger}{ledger}
\umlinherit{delete-ledger}{ledger}
\umlinherit{add-transaction}{transaction}
\umlinherit{delete-transaction}{transaction}
\umlinherit{edit-transaction}{transaction}
\umlinclude{ledger}{login}
\umlinclude{transaction}{add-ledger}
\end{tikzpicture}

\subsection{Käyttäjäryhmät}

\begin{description}
  \item[käyttäjä] järjestelmään rekisteröitynyt käyttäjä
\end{description}

\subsection{Käyttötapauskuvaukset}

\begin{description}[style=nextline]
  \item[tilikirjojen käsittely]{
      Kirjautunut käyttäjä voi lisätä ja poistaa tilikirjoja. Kun tilikirjan
      poistaa, kaikki siihen liittyvät tilitapahtumat, osapuolet ja valuutat
      poistuvat samalla.
    }
  \item[tilitapahtumien käsittely]{
      Kirjautunut käyttäjä voi lisätä, poistaa ja muokata tietyn tilikirjan
      tilitapahtumia. Tilitapahtumaan liittyy osapuolia sekä näiden
      tilitapatumaan liittyviä menoja ja tuloja. Menoja ja tuloja voidaan
      kirjata eri valuutoilla ja tilitapahtumien kokonaissummaa voidaan
      tarkastella haluttuun valuuttaan muunnettuna.
    }
  \item[valuuttakurssien muokkaus]{
      Kirjautunut käyttäjä voi muokata tilikirjassa olevien valuuttojen
      kursseja.
    }
  \item[loppusaldojen tulostus]{
      Kirjautunut käyttäjä voi tulostaa tilikirjan loppusaldot kaikki
      osapuolten osalta.
    }
  \item[maksusuunnitelman tulostus]{
      Kirjautunut käyttäjä pyytää järjestelmältä ehdotusta velkojen
      takaisinmaksujärjestelystä.
    }
  \end{description}

\bibliographystyle{plainnat}
\bibliography{dokumentaatio}
\end{document}
