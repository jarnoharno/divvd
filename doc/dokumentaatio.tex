\documentclass[a4paper,parskip=half]{scrartcl}
\usepackage[finnish]{babel}
\usepackage[utf8]{inputenc}
\usepackage[T1]{fontenc}
\usepackage{mathptmx}
\usepackage[numbers]{natbib}
\usepackage{hyperref}
\usepackage{tikz-uml}
\usepackage{enumitem}
\usepackage{calc}
\usepackage{graphicx}
\usepackage{fancyvrb}
\setlist[description]{font={\rmfamily}}
\addtokomafont{disposition}{\rmfamily}

\subject{Aineopintojen harjoitustyö: Tietokantasovellus}
\author{Jarno Leppänen}
\title{Divvd -- Useilla valuutoilla toimiva kustannusten jako}
\subtitle{Dokumentaatio}
\date{27.4.2014}

\begin{document}

\maketitle

\tableofcontents

\section{Johdanto}

Kustannusten jako on tavallinen ongelma illanvietoissa ystävien
kanssa tai taloudenpidossa puolison tai kämppäkavereiden välillä. Erityisen
hankalaksi tehtävä osoittautuu, jos jaettavissa kustannuksissa on käytetty
useita eri valuuttoja. Ulkomailla matkaillessa syntyy helposti tilanteita,
joissa esimerkiksi matkaliput ja majoitukset on maksettu kotimaan valuutassaa,
mutta juoksevat kulut hoidetaan kohdemaan valuutassa. Miten kustannukset
tulisi tällöin jakaa?

Työssä toteutetaan kustannusten jakoon tarkoitettu kirjanpitojärjestelmä, jossa
kirjauksia voi tehdä eri valuutoilla. Käyttäjä voi rekisteröitymisen ja
kirjautumisen jälkeen luoda \textit{tilikirjoja}, jotka ovat kokoelmia
käyttäjän syöttämistä \textit{tilitapahtumista}. Tilitapahtumaan kirjataan
tapahtuman osapuolet sekä näiden tulot ja menot tapahtumaan liittyen.
Järjestelmä laskee tilikirjassa olevien osapuolten kokonaissaldot haluttuun
valuuttaan muunnettuna ja osaa lisäksi ehdottaa tapoja velkojen
takaisinmaksuun. Takaisinmaksujärjestely voidaan ratkaista esimerkiksi
minimoimalla järjestelyssä tapahtuvien transaktioiden
lukumäärä\cite{verhoeff2004settling}.

Palautetussa työssä takaisinmaksujärjestelyn laatimista ei ole toteutettu.

\subsection{Toimintaympäristö}

Ohjelma toteutetaan web-sovelluksena heroku-PaaS-palvelun päälle.
Sovelluskehyksenä käytetään herokun tukemaa node.js-ympäristöä.
Selainkäyttöliittymän sovelluskehyksenä käytetään AngularJS-kirjastoa.
Ohjelmointikielenä sekä selain-, että palvelinympäristöissä on Javascript.

Tietokantana toimii herokun tarjoama PostgreSQL. Tietokantaa käsitellään
suoraan Postgre\-SQL:n laajennetun SQL-kielen avulla. Toteutuksessa
hyödynnetään SQL-standardiin kuulumattomia trigger- ja view-mekanismeja, joten
tietokantamoduli on sidottu Post\-gre\-SQL-jär\-jes\-tel\-mään.

\section{Yleiskuva järjestelmästä}

\subsection{Käyttötapauskaavio}

\begin{center}
\newcommand{\coldist}{5}
\begin{tikzpicture}
\begin{umlsystem}[x=5,fill=yellow!20]{Divvd-järjestelmä}
  \umlusecase[name=register]{rekisteröityminen}
  \umlusecase[y=-2,name=login]{kirjautuminen}
  \umlusecase[y=-4,name=ledger,width=1.5cm]{tilikirjojen käsittely}
  \umlusecase[y=-6,name=transaction,width=2.2cm]{tilitapahtumien käsittely}
  \umlusecase[y=-10,name=totals,width=2cm]{loppusaldojen tulostus}
  \umlusecase[y=-12,name=settle,width=2.9cm]{maksusuunnitelman tulostus}
  \umlusecase[y=-8,name=currency,width=2.4cm]{valuuttakurssien muokkaus}
  \umlusecase[y=-4,x=\coldist,name=add-ledger,width=1.5cm]{tilikirjan lisäys}
  \umlusecase[y=-2,x=\coldist,name=delete-ledger,width=1.5cm]{tilikirjan poisto}
  \umlusecase[y=-6,x=\coldist,name=add-transaction,width=2.2cm]{tilitapahtuman
    lisäys}
  \umlusecase[y=-10,x=\coldist,name=delete-transaction,width=2.2cm]{tilitapahtuman
    poisto}
  \umlusecase[y=-8,x=\coldist,name=edit-transaction,width=2.2cm]{tilitapahtuman
    muokkaus}
\end{umlsystem}
\umlactor[y=-6,x=-1,name=user]{käyttäjä}
\umlassoc{user}{register}
\umlassoc{user}{login}
\umlassoc{user}{ledger}
\umlassoc{user}{transaction}
\umlassoc{user}{totals}
\umlassoc{user}{settle}
\umlassoc{user}{currency}
\umlinherit{add-ledger}{ledger}
\umlinherit{delete-ledger}{ledger}
\umlinherit{add-transaction}{transaction}
\umlinherit{delete-transaction}{transaction}
\umlinherit{edit-transaction}{transaction}
\umlinclude{ledger}{login}
\umlinclude{transaction}{add-ledger}
\end{tikzpicture}
\end{center}

\subsection{Käyttäjäryhmät}

\begin{description}
  \item[käyttäjä] järjestelmään rekisteröitynyt käyttäjä
\end{description}

\subsection{Käyttötapauskuvaukset}

\begin{description}[style=nextline]
  \item[tilikirjojen käsittely]{
      Kirjautunut käyttäjä voi lisätä ja poistaa tilikirjoja. Kun tilikirjan
      poistaa, kaikki siihen liittyvät tilitapahtumat, osapuolet ja valuutat
      poistuvat samalla.
    }
  \item[tilitapahtumien käsittely]{
      Kirjautunut käyttäjä voi lisätä, poistaa ja muokata tietyn tilikirjan
      tilitapahtumia. Tilitapahtumaan liittyy osapuolia sekä näiden
      tilitapatumaan liittyviä menoja ja tuloja. Menoja ja tuloja voidaan
      kirjata eri valuutoilla ja tilitapahtumien kokonaissummaa voidaan
      tarkastella haluttuun valuuttaan muunnettuna.
    }
  \item[valuuttakurssien muokkaus]{
      Kirjautunut käyttäjä voi muokata tilikirjassa olevien valuuttojen
      kursseja.
    }
  \item[loppusaldojen tulostus]{
      Kirjautunut käyttäjä voi tulostaa tilikirjan loppusaldot kaikki
      osapuolten osalta.
    }
  \item[maksusuunnitelman tulostus]{
      Kirjautunut käyttäjä pyytää järjestelmältä ehdotusta velkojen
      takaisinmaksujärjestelystä. Tämä ominaisuus on toteuttamatta lopullisessa
      järjestelmässä.
    }
\end{description}

\section{Järjestelmän tietosisältö}

\subsection{Käsitekaavio}

Oheiseen kaavioon ei ole merkitty käyttöliittymän tarvitsemia viitteitä
\texttt{Currency}-tietokohteeseen. Kaikki muut tietokohteet sisältävät
näitä viitteitä erilaisten aggregaattien näyttämistä varten.

\begin{center}
\begin{tikzpicture} 
  \umlemptyclass{User} 
  \umlemptyclass[y=-3]{Ledger}
  \umlemptyclass[y=-3,x=4]{Person}
  \umlemptyclass[y=-6]{Transaction}
  \umlemptyclass[y=-6,x=4]{Amount}
  \umlemptyclass[y=-9]{Currency}
  \umlaggreg[mult1=1..*,mult2=*]{User}{Ledger}
  \umlunicompo[mult2=*]{Ledger}{Person}
  \umlunicompo[mult2=*]{Ledger}{Transaction}
  \umlunicompo[mult2=*]{Transaction}{Amount}
  \umlunicompo[mult2=1]{Person}{Amount}
  \umlunicompo[geometry=-|-,mult2=*,arm1=-2cm]{Ledger}{Currency}
  \umlassoc[geometry=-|,mult1=0..1,mult2=*]{User}{Person}
  \umlassoc[mult1=*,mult2=*]{Transaction}{Person}
  \umluniassoc[geometry=|-,mult1=*,mult2=1]{Amount}{Currency}
\end{tikzpicture}
\end{center}

Tilitapahtuman ylijäämä jaetaan kaikkien tilitapahtumaan
liittyvien henkilöiden kesken. \texttt{Transaction}- ja
\texttt{Person}-tietokohteiden välinen monesta moneen -suhde kuvaa tätä
relaatiota.

\subsection{Tietokohteet}

\subsubsection{User}
\begin{tabular}{ | l | l | l | }
  \hline
  \textbf{Attribuutti} & \textbf{Arvojoukko} & \textbf{Kuvailu} \\ \hline
  Username & Merkkijono & Käyttäjän nimi \\ \hline
  Role & Enumeraatio & Käyttäjän rooli \\ \hline
  Hash & 256-bittinen binäärinen merkkijono &
    Salasanasta laskettu hajautusarvo
    \\ \hline
  Salt & 64-bittinen binäärinen merkkijono & Satunnainen binäärinen merkkijono
    \\ \hline
\end{tabular}

\subsubsection{Ledger}
\begin{tabular}{ | l | l | l | }
  \hline
  \textbf{Attribuutti} & \textbf{Arvojoukko} & \textbf{Kuvailu} \\ \hline
  Title & Merkkijono & Tilikirjan nimi \\ \hline
\end{tabular}

\subsubsection{Currency}
\begin{tabular}{ | l | l | l | }
  \hline
  \textbf{Attribuutti} & \textbf{Arvojoukko} & \textbf{Kuvailu} \\ \hline
  Code & Merkkijono & Valuutan nimi tai lyhenne \\ \hline
  Rate & Desimaaliluku & Valuutan kurssi \\ \hline
\end{tabular}

\subsubsection{Transaction}
\begin{tabular}{ | l | l | l | }
  \hline
  \textbf{Attribuutti} & \textbf{Arvojoukko} & \textbf{Kuvailu} \\ \hline
  Date & Päivämäärä & Tilitapahtuman päivämäärä \\ \hline
  Description & Merkkijono & Selite \\ \hline
  Location & Merkkijono & Tilitapahtuman paikka \\ \hline
  Type & Merkkijono & Tilitapahtuman tyyppi \\ \hline
  Transfer & Totuusarvo & Totuusarvo, joka kertoo, onko tilitapahtuma
    rahansiirto vai kulu \\ \hline
\end{tabular}

\subsubsection{Person}
\begin{tabular}{ | l | l | l | }
  \hline
  \textbf{Attribuutti} & \textbf{Arvojoukko} & \textbf{Kuvailu} \\ \hline
  Name & Merkkijono & Osapuolen nimi \\ \hline
\end{tabular}

\subsubsection{Amount}
\begin{tabular}{ | l | l | l | }
  \hline
  \textbf{Attribuutti} & \textbf{Arvojoukko} & \textbf{Kuvailu} \\ \hline
  Amount & Desimaaliluku & Rahamäärä \\ \hline
\end{tabular}

\section{Relaatiotietokantakaavio}

Jokaisella tietokannan taululla on synteettinen primäärinen kokonaislukuavain
nimellä \textit{taulu}\_id. Näitä avaimia ei ole piirretty kaavioon.

Kaavion tietokohteiden kuvauksissa keskimmäiseen laatikkoon on nimetty
kaikki vierasavaimet, jotka ovat aina kokonaislukutyyppisiä. Taulun muut
sarakkeet tietotyyppeineen on merkitty alimpaan laatikkoon.

Kaavioon ei ole merkitty käyttöliittymässä ja saldojen laskennassa tarvittavia
viittauksia \texttt{currency}-tauluun. Näiden relaatioiden vierasavaimet on
merkitty kaavioon kursiivilla.

\begin{center}
\begin{tikzpicture} 
  \umlclass[y=-4,x=-6]{user}{ 
  }{
    username: text \\
    role: role \\
    hash: bytea \\
    salt: bytea
  } 
  \umlclass[y=-4,x=-2]{owner}{
    user\_id \\
    ledger\_id \\
    \umlvirt{currency\_id} \\
    \umlvirt{\umlvirt{total\_credit\_currency\_id}} \\
  }{
  }
  \umlclass[y=-8,x=-2]{ledger}{
  }{
    title: text
  }
  \umlclass[y=-8,x=-6]{ledger\_settings}{
    ledger\_id \\
    \umlvirt{total\_currency\_id} \\
  }{
  }
  \umlclass[y=-8,x=2]{person}{
    \umlvirt{currency\_id} \\
    user\_id \\
    ledger\_id
  }{
    name: text \\
  }
  \umlclass[y=-15,x=-2]{transaction}{
    ledger\_id \\
    \umlvirt{currency\_id} \\
  }{
    description: text \\
    date: timestamp \\
    type: text \\
    location: text \\
    transfer: boolean \\
  }
  \umlclass[y=-15,x=-7]{owner\_transaction\_settings}{
    owner\_id \\
    transaction\_id \\
    \umlvirt{owner\_balance\_currency\_id} \\
    \umlvirt{total\_value\_currency\_id} \\
    \umlvirt{owner\_total\_credit\_currency\_id} \\
  }{
  }
  \umlclass[y=-15,x=2]{participant}{
    \umlvirt{currency\_id} \\
    transaction\_id \\
    person\_id \\
  }{
  }
  \umlclass[y=-20,x=-2]{amount}{
    \umlvirt{currency\_id} \\
    transaction\_id \\
    person\_id
  }{
    amount: numeric \\
  }
  \umlclass[y=-11,x=-6]{currency}{
    ledger\_id
  }{
    code: text \\
    rate: numeric \\
  }
  \umluniassoc{owner}{user}
  \umluniassoc{owner}{ledger}
  \umluniassoc{person}{ledger}
  \umluniassoc[geometry=|-|,weight=1.5]{person}{user}
  \umluniassoc[mult1=1]{ledger\_settings}{ledger}
  \umluniassoc[geometry=|-|,anchor1=140,,anchor2=-140,weight=0.8]
    {owner\_transaction\_settings}{owner}
  \umluniassoc{owner\_transaction\_settings}{transaction}
  \umluniassoc{transaction}{ledger}
  \umluniassoc{participant}{transaction}
  \umluniassoc{participant}{person}
  \umluniassoc{amount}{transaction}
  \umluniassoc[geometry=-|-,weight=1.5]{amount}{person}
  \umluniassoc[geometry=-|,anchor2=-120]{currency}{ledger}
\end{tikzpicture}
\end{center}

\section{Järjestelmän yleisrakenne}

Projektin palvelinosio noudattaa MVC-arkkitehtuuria. Koska palvelin tarjoaa
vain JSON-rajapinnan, toimii "view"-komponentti automaattisesti alla olevassa
sovelluskehyksessä, eikä lähdekoodihakemistopuussa ole "view"-kansiota.
Sovelluskehyksen tietokantakomponentti muuntaa tietokannasta saadut vastaukset
javascript-olioiksi, jotka muunnetaan edelleen automaattisesti
HTTP-vastauksissa oleviksi merkkijonoiksi.

\subsection{Hakemistorakenne}

\begin{description}[style=nextline]
  \item[/]{
      Projektin juuri sisältää kaikki projektin kehityksessä tarvittavat
      tiedostot.
    }
  \item[/db]{
      SQL-tietokantalistaukset.
    }
  \item[/doc]{
      \LaTeX-muotoinen dokumentaatio.
    }
  \item[/local]{
      Paikallisessa sovelluskehityksessä tarvittavat apuohjelmat.
    }
  \item[/test]{
      Testitiedostot.
    }
  \item[/build]{
      Väliaikaiset käännöstiedostot, jotka eivät kuulu versionhallintaan.
    }
  \item[/app]{
      Tuotantopalvelimella tarvittavat tiedostot.
    }
  \item[/app/controllers]{
      Kontrollerifunktiot, jotka käsittelevät käyttäjien palvelupyynnöt.
    }
  \item[/app/lib]{
      Palvelimen apukirjastot. Sisältää esimerkiksi autentikointiin ja
      tietokantayhteyteen liittyvät moduulit.
    }
  \item[/app/models]{
      Palvelimen domain-oliot.
    }
  \item[/app/dao]{
      Domain-olioiden persistointi tietokantaan.
    }
  \item[/app/public]{
      Palvelimelta saatavat staattiset tiedostot, selaimessa ajettava
      asiakassovellus.
    }
  \item[/app/public/secure]{
      Vain TLS-yhteyden yli lähetettävät tiedostot.
    }
  \item[/app/public/nonsecure]{
      Salatun ja salaamattoman yhteyden yli lähetettävät tiedostot.
    }
  \item[/app/public/nonsecure/css]{
      Asiakassovelluksen tyylitiedostot.
    }
  \item[/app/public/nonsecure/doc]{
      Dokumentaatiotiedostot.
    }
  \item[/app/public/nonsecure/img]{
      Asiakassovelluksen kuvatiedostot.
    }
  \item[/app/public/nonsecure/partials]{
      Asiakassovelluksen template-tiedostot
    }
  \item[/app/public/nonsecure/js]{
      Asiakassovelluksen javascript-koodi.
    }
  \item[/app/public/nonsecure/js/controllers]{
      Asiakassovelluksen kontrollerit.
    }
  \item[/app/public/nonsecure/js/services]{
      Asiakassovelluksen palvelumoduulit.
    }
\end{description}

\section{Käyttöliittymä ja järjestelmän komponentit}

Web-käyttöliittymä toteutetaan "single page"-applikaationa, jossa selaimen
lataama javascript-käyttöliittymä keskustelee palvelimen tarjoaman
JSON-rajapinnan kanssa ja päivittää selaimen osoitekenttää kulloinkin
näytettävää resurssia vastaavaksi. Arkkitehtuurin ansiosta yksittäinen
käyttöliittymänäkymä voi päivittää tietokantaa osissa ilman, että näkymää
tarvitsee ladata palvelimelta kokonaan uudelleen.

\subsection{Käyttöliittymäkaavio}

\begin{tikzpicture} 
\begin{umlstate}[name=guest]{Vierailija} 
\umlbasicstate[name=front]{/}
\umlbasicstate[name=login,x=3]{/login}
\umlbasicstate[name=signup,x=6]{/signup}
\umltrans{front}{login} 
\umltrans{login}{front} 
\umltrans{signup}{login} 
\umltrans{login}{signup} 
\umltrans[geometry=|-|,arm1=-1.5cm]{front}{signup}
\umltrans[geometry=|-|,arm2=-1.5cm]{signup}{front}
\end{umlstate} 

\begin{umlstate}[y=-6,name=member]{Kirjautunut käyttäjä} 
\umlbasicstate[name=ledgers,y=0,x=1]{/,/ledgers}
\umlbasicstate[name=ledger,y=-3,x=1]{/ledgers/:ledger\_id}
\umlbasicstate[name=summary,y=0,x=8]{/ledgers/:ledger\_id/summary}
\umlbasicstate[name=currencies,y=-3,x=8]{/ledgers/:ledger\_id/currencies}
\umlbasicstate[name=persons,y=-6,x=8]{/ledgers/:ledger\_id/persons}
\umlbasicstate[name=transaction,y=-9,x=2]
  {/ledgers/:ledger\_id/transactions/:transactions\_id}
\umltrans[geometry=-|-,anchor1=10,anchor2=0,arm2=0.23cm]{transaction}{persons}
\umltrans[geometry=-|-,anchor1=0,anchor2=0,arm2=0.4cm]{transaction}{currencies}
\umltrans[geometry=-|-,anchor1=-10,anchor2=0,arm2=0.8cm]{transaction}{summary}
\end{umlstate}

\umltrans[geometry=|-|,anchor1=-130,anchor2=130]{front}{ledgers}
\umltrans[geometry=|-|,anchor1=130,anchor2=-130]{ledgers}{front}
\umltrans[geometry=|-|,weight=0.4]{login}{ledgers}
\umltrans[geometry=|-|,anchor1=-50,anchor2=50]{signup}{ledgers}
\umltrans{ledgers}{ledger}
\umltrans{ledger}{ledgers}
\umltrans[geometry=|-|]{ledger}{transaction}
\umltrans[geometry=|-|]{transaction}{ledger}
\umltrans[geometry=|-,anchor1=163,anchor2=180]{transaction}{ledgers}
\umltrans{ledgers}{summary}
\umltrans{summary}{ledgers}
\umltrans[geometry=-|-,anchor1=-20,anchor2=165]{ledgers}{currencies}
\umltrans[geometry=-|-,anchor2=-20,anchor1=165]{currencies}{ledgers}
\umltrans[geometry=-|-,anchor1=-30,anchor2=165,weight=0.4]{ledgers}{persons}
\umltrans[geometry=-|-,anchor2=-30,anchor1=165,weight=0.6]{persons}{ledgers}
\umltrans[geometry=-|-,anchor1=10,anchor2=-165,weight=0.45]{ledger}{summary}
\umltrans{ledger}{currencies}
\umltrans[geometry=-|-,anchor1=-10,weight=0.3]{ledger}{persons}
\end{tikzpicture}

\subsection{Näkymät}

\begin{description}[style=nextline]
  \item[/]{
      Etusivu näyttää kirjautumattomalle vierailijalle kirjautumislomakkeen sekä
      linkin rekisteröitymissivulle. Kirjautunut käyttäjä näkee listan omista
      tilikirjoistaan.
    }
  \item[/login]{
      Kirjautumissivu, joka näyttää pelkän lomakkeen. Onnistuneen kirjautumisen
      jälkeen käyttäjä ohjataan kirjautuneena etusivulle.
    }
  \item[/signup]{
      Rekisteröitymissivu. Onnistuneen rekisteröinnin jälkeen käyttäjä ohjataan
      kirjautuneena etusivulle.
    }
  \item[/ledgers]{
      Tilikirjojen listaussivu näyttää rekisteröityneen käyttäjän tilikirjat.
      Listaussivun kautta tilikirjoja voi lisätä tai poistaa. Sivulta voi myös
      siirtyä tarkastelemaan yksittäisen tilikirjan tilitapahtumia.
    }
  \item[/ledgers/:ledger\_id]{
      Tilikirjan tilitapahtumien listaussivu näyttää kaikki tilikirjaan
      liittyvät tilitapahtumat. Sivu voidaan näyttää vain, jos kirjautunut
      käyttää omistaa tilikirjan. Sivun avulla tilitapahtumia voi lisätä ja
      poistaa. Sivun kautta pääsee myös muokkaamaan yksittäistä tilitapahtumaa.
    }
  \item[/ledgers/:ledger\_id/summary]{
      Tilikirjan yhteenveto listaa kaikkien tilikirjaan liittyvien osapuolten
      kokonaissaldot sekä järjestelmän automaattisesti tuottaman
      velanmaksusuunnitelman, jonka voi halutessaan siirtää tilitapahtumaksi,
      jolloin tilikirjan saldot tasapainottuvat.
    }
  \item[/ledgers/:ledger\_id/currencies]{
      Valuuttojen muokkaussivu näyttää tilikirjaan liittyvät valuutat
      kursseineen.  Sivun avulla valuuttoja voi lisätä, poistaa ja muokata.
    }
  \item[/ledgers/:ledger\_id/persons]{
      Henkilösivu näyttää tilikirjaan osallistuvat henkilöt. Sivun avulla
      henkilöitä voi lisätä, poistaa ja muokata.
    }
  \item[/ledgers/:ledgers\_id/transactions/:transaction\_id]{
      Tilitapahtumasivulla tilitapahtumaa voi muokata. Tilitapahtumaan voi
      lisätä osapuolten tekemiä maksuja ja velkasitoumuksia.
    }
\end{description}

\section{Asennustiedot}

\subsection{Kehitysversio}

Sovelluksen paikallisen kehitysversion ajamiseen tarvitaan
node.js-sovelluskehys, PostgreSQL-tietokanta sekä openSSH-ohjelmisto. Kaikki
sovelluksen kääntämiseen, päivittämiseen ja tietokannan käsittelyyn
liittyvät toiminnot hoidetaan make-ohjelmalla. Make myös huolehtii kaikista
riippuvuuksista. Kaikki väliaikaiset käännöstiedostot sekä
paikallinen tietokanta sijoitetaan juuressa olevaan build-hakemistoon.

Tietokannan voi alustaa seuraavalla komennolla:
\begin{Verbatim}
make reset-db
\end{Verbatim}

Sovelluksen voi ajaa paikallisesti seuraavalla komennolla:
\begin{Verbatim}
make run
\end{Verbatim}
Tämä huolehtii tietokannan alustamisesta ja salausavainten generoinnista,
käynnistää tietokantaprosessin, https-proxypalvelimen sekä varsinaisen
sovelluspalvelimen ja jää odottamaan käyttäjän keskeytystä (Ctrl-C).
Sovellus näkyy osoitteissa \url{https://localhost:4000} sekä
\url{http://localhost:3000}.

\subsection{Dokumentaatio}
Dokumentaation kääntämisessä käytetään pdflatex-ohjelmistoa, latexmk-skriptiä
sekä muokattua TikzUML-pakettia (\url{https://github.com/jlep/tikzuml}).
Pelkän dokumentaation voi kääntää ja asentaa seuraavalla komennolla:
\begin{Verbatim}
make doc
\end{Verbatim}

Dokumentaatiota voi muokata reaaliajassa seuraavalla komennolla:
\begin{Verbatim}
make doc-cont
\end{Verbatim}
Tämä jää odottamaan muutoksia dokumentaation lähdekoodissa ja kääntää
dokumentaation tarvittavilta osin uudelleen muutoksen havaitessaan. Prosessin
voi keskeyttää näppäinkomennolla Ctrl-C.

\subsection{Tuotantoversio}

Sovellus on suunniteltu Heroku-alustalle. Tuotantoversion asentaminen
vaatii heroku toolbelt -ohjelmiston. Tietokannan yhteystiedot ja salasana
on tallennettu heroku-tilille ympäristömuuttujina, eivätkä siis ole näkyvillä
missään paikallisesti.

Tuotantoversion voi päivittää työntämällä app-hakemistoon alustetun
git-repositorion herokuun.

Herokussa olevan tietokannan voi alustaa seuraavalla komennolla:
\begin{Verbatim}
make heroku-init-data
\end{Verbatim}

\section{Käynnistys- / käyttöohje}
Sovellus on asennettu osoitteeseen \url{https://divvd.herokuapp.com/}.
Sovellukseen voi rekisteröidä uuden käyttäjätunnuksen tai kirjautua
testitunnuksilla test/test.

\section{Testaus, tunnetut bugit ja puutteet sekä jatkokehitysideat}

Hakemisto \texttt{test} sisältää testit käyttäjähallintaan liittyville
JSON-rajapintaosoitteille. Näiden testien lisäksi projektia ei muuten
ole testattu.

Tietokanta on mahdollista saada epäkonsistenttiin tilaan JSON-rajapinnan
kautta esimerkiksi tuhoamalla käytössä olevia valuuttoja. Tällaisia tilanteita
ei ole testattu millään tavalla. Epäloogista tilaa ei kuitenkaan pitäisi olla
mahdollista saavuttaa web-käyttöliittymän kautta.

Asiakassovellus ei huomaa, jos asiakkaan sessio jostain syystä häviää
palvelimelta. Virhetilasta selviää lataamalla sivun uudestaan selaimen
refresh-toiminnolla.

\subsection{Jatkokehitysideat}

\begin{itemize}
  \item Rajapinnan testaus ja bugien korjaus
  \item Rajapinnan versiointi
  \item Kirjautuminen HTTP-otsakkeessa olevalla käyttäjäkohtaisella avaimella
  \item Sessioavainten tallentaminen esimerkiksi Redis-tietokantaan
  \item Offline-käyttö esimerkiksi WebSQL:n avulla sekä automaattinen
    synkronointi palvelimelle
  \item Käyttöliittymät mobiilialustoille esimerkiksi Android- tai iOS-
    applikaatioina
  \item Intuitiivisempi ja virtaviivaisempi web-käyttöliittymä
  \item Uutta analytiikkaa käyttäjälle tilikirjoista kaavioiden ja karttojen
    avulla
  \item Rajatut palvelut rekisteröitymättömälle käyttäjälle
  \item OpenAuth-tunnistautuminen
  \item Velkojen takaisinmaksusuunnitelman laskenta.
\end{itemize}

\section{Omat kokemukset}

Harjoitustyö oli varsin laaja ja toteutus vei paljon aikaa. Kurssin laajuus
ei välttämättä ole kovin perusteltu, jos tarkoituksena on opetella nimenomaan
tietokantaohjelmointia. Web-ohjelmoinnin ja käyttöliittymän rakentaminen
saattaa viedä enemmän aikaa kuin varsinainen tietokantaohjelmointi.
Web-ohjelmoinnin näkökulmasta ORM-kirjastojen käyttökielto on varsin
ristiriitainen: mitään web-sovellusta tuskin rakennetaan nykyään ilman
ORM-komponenttia.

Itse opin paljon uutta HTTP-protokollasta, REST-rajapintojen tekemisestä,
autentikoinnista ja angular-sovelluskehyksestä, mutta myös PostgreSQL-
tietokannasta, triggereistä, transaktioista ja näkymistä. Mielenkiintoista
olisi vielä nähdä, miten suunnittelemani tietokanta skaalautuu suurella
datamäärällä.

\section{Muu dokumentaatio}

Tietokantaskeeman alustuslistaus on nähtävillä tiedostossa
\texttt{db/init-schema.sql}.

\section{JSON-rajapinta}

Web-käyttöliittymä keskustelee tietokannan kanssa palvelimen tarjoaman
JSON-rajapinnan kautta. Rajapinnan avulla järjestelmän laajentaminen muille
alustoille on yksinkertaista. JSON-rajapinta on julkinen, joten myös kolmannet
osapuolet voivat halutessaan hyödyntää sitä.

Rajapinnan suunnittelussa on käytetty apuna Kippt-palvelun API-dokumentaatiota,
joka on nähtävillä osoitteessa
\url{https://github.com/kippt/api-documentation/}.

\subsection{Formaatti}

Rajapinta hyväksyy ainoastaan JSON-muotoiset POST-parametrit ja myös vastaa
ainoastaan JSON-muodossa. POST-palvelupyyntöihin on liitettävä
\texttt{Content-Type: application/json} -otsake.

\subsection{Tunnistautuminen}

Rajapinta vaatii tunnistautumista joko HTTP Basic -autentikoinnin tai
selainevästeiden avulla. Selaimen uloskirjautumisen mahdollistamiseksi
kirjautumiseen liittyvät osoitteet /api/login, /api/logout, /api/signup ja
/api/account eivät hyväksy HTTP Basic -autentikointia. Kaikki palveluosoitteet
lähettävät vain salatun yhteyden yli lähetettävän HttpOnly-sessioevästeen,
jonka avulla käyttäjän kirjautumisen tilaa seurataan.

Rajapinta vastaa tunnistautumattomaan pyyntöön statuskoodilla \texttt{401
Unauthorized}. Vastauksen WWW-Authenticate -otsakkeen voi halutessaan muuttaa
Custom-tyyppiseksi liittämällä palvelupyyntöön GET-parametri auth=hidden. Näin
voidaan estää selaimen automaattisesti näyttämä kirjautumisikkuna.

Kirjautumista vaativat palveluosoitteet palauttavat \texttt{401}-virheviestin
mukana \texttt{WWW-Authenticate: Basic realm="Authorization Required"}
-otsakkeen. Selaimesta kutsuttuna tämä otsake aiheuttaa yleensä
kirjautumisikkunan avautumisen. Tämän voi estää liittämällä kutsuun
GET-parametriksi \texttt{auth=hidden}, jolloin vastauksen otsake muuttuu
muotoon \texttt{WWW-Authenticate: Custom realm="Authorization Required"}.
Aiemmin mainitut selaimen kirjautumiseen liittyvät osoitteet palauttavat aina
custom-tyyppisen otsakkeen.

\subsection{Virhetilanteet}

Asiakkaan aiheuttamiin virhetilanteisiin vastataan statuskoodeilla \texttt{400
Bad request}, jos syötteessä tai sovelluksen tilassa on virhe. Virheellisiin
palveluosoitteisiin vastataan statuskoodilla \texttt{404 Not found}. Useimpiin
osoitevirheisiin vastataan kuitenkin lähettämällä asiakkaalle
sovellusapplikaation etusivu. Tämä on välttämätöntä single-page
-arkkitehtuurissa asiakkaan reitityksen toimivuuden takia. Muihin kuin
GET-tyyppisiin pyyntöihin, jotka on lähetetty tunnistamattomaan osoitteeseen
vastataan statuskoodilla \texttt{405 Not allowed}.

Palvelimen sisäiseen virhetilanteeseen vastataan statuskoodilla \texttt{500
Internal server error}. Sisäisen virhevastauksen mukana tulevan viestin sisältö
on aina "server error", jottei palvelun toteutuksen yksityiskohdista anneta
tietoja käyttäjälle.

\bibliographystyle{plainnat}
\bibliography{dokumentaatio}
\end{document}
